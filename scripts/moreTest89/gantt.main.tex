\documentclass[landscape]{article}
\usepackage{gantt}
\pagestyle{empty}
\begin{document}
\begin{figure}
\begin{gantt}[fontsize=\scriptsize]{32}{44}
\begin{ganttitle}
\numtitle{0}{1}{43}{1}
\end{ganttitle}
\ganttbar{  call void @platform\_main\_begin() #3}{2}{1}
\ganttbar{  call void @crc32\_gentab() #3}{4}{1}
\ganttbar{  \%1 = call i32 @func\_1() #3}{6}{1}
\ganttbar{  \%2 = load i32* @g\_3, align 4}{7}{2}
\ganttcon{9}{4}{9}{5}
\ganttbar{  \%3 = zext i32 \%2 to i64}{9}{1}
\ganttcon{10}{5}{10}{6}
\ganttbar{  call void @transparent\_crc(i64 \%3, i8* getelementptr inbounds ([4 x i8]* @.s...}{10}{1}
\ganttbar{  \%4 = load volatile i8* @g\_51, align 1}{11}{2}
\ganttcon{13}{7}{13}{8}
\ganttbar{  \%5 = zext i8 \%4 to i64}{13}{1}
\ganttcon{14}{8}{14}{9}
\ganttbar{  call void @transparent\_crc(i64 \%5, i8* getelementptr inbounds ([5 x i8]* @.s...}{14}{1}
\ganttbar{  \%6 = load i32* @g\_54, align 4}{15}{2}
\ganttcon{17}{10}{17}{11}
\ganttbar{  \%7 = sext i32 \%6 to i64}{17}{1}
\ganttcon{18}{11}{18}{12}
\ganttbar{  call void @transparent\_crc(i64 \%7, i8* getelementptr inbounds ([5 x i8]* @.s...}{18}{1}
\ganttbar{  \%8 = load i16* @g\_55, align 2}{19}{2}
\ganttcon{21}{13}{21}{14}
\ganttbar{  \%9 = sext i16 \%8 to i64}{21}{1}
\ganttcon{22}{14}{22}{15}
\ganttbar{  call void @transparent\_crc(i64 \%9, i8* getelementptr inbounds ([5 x i8]* @.s...}{22}{1}
\ganttbar{  \%10 = load i32* @g\_59, align 4}{23}{2}
\ganttcon{25}{16}{25}{17}
\ganttbar{  \%11 = sext i32 \%10 to i64}{25}{1}
\ganttcon{26}{17}{26}{18}
\ganttbar{  call void @transparent\_crc(i64 \%11, i8* getelementptr inbounds ([5 x i8]* @....}{26}{1}
\ganttbar{  \%12 = load i16* @g\_81, align 2}{27}{2}
\ganttcon{29}{19}{29}{20}
\ganttbar{  \%13 = zext i16 \%12 to i64}{29}{1}
\ganttcon{30}{20}{30}{21}
\ganttbar{  call void @transparent\_crc(i64 \%13, i8* getelementptr inbounds ([5 x i8]* @....}{30}{1}
\ganttbar{  \%14 = load i32* @g\_119, align 4}{31}{2}
\ganttcon{33}{22}{33}{23}
\ganttbar{  \%15 = sext i32 \%14 to i64}{33}{1}
\ganttcon{34}{23}{34}{24}
\ganttbar{  call void @transparent\_crc(i64 \%15, i8* getelementptr inbounds ([6 x i8]* @....}{34}{1}
\ganttbar{  \%16 = load i16* @g\_123, align 2}{35}{2}
\ganttcon{37}{25}{37}{26}
\ganttbar{  \%17 = sext i16 \%16 to i64}{37}{1}
\ganttcon{38}{26}{38}{27}
\ganttbar{  call void @transparent\_crc(i64 \%17, i8* getelementptr inbounds ([6 x i8]* @....}{38}{1}
\ganttbar{  \%18 = load i32* @crc32\_context, align 4}{39}{2}
\ganttcon{41}{28}{41}{29}
\ganttbar{  \%19 = xor i32 \%18, -1}{41}{1}
\ganttcon{42}{29}{42}{30}
\ganttbar{  \%20 = call i32 @platform\_main\_end(i32 \%19, i32 0) #3}{42}{1}
\ganttcon{43}{30}{43}{31}
\ganttbar{  ret i32 \%20}{43}{1}
\end{gantt}
\end{figure}
\end{document}
